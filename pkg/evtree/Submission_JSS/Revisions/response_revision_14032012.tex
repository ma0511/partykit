\documentclass[DIN, pagenumber=false, parskip=half,% 
               fromalign=left, fromphone=true,%  
               fromemail=true, fromurl=false, %
               fromlogo=true, fromrule=false, fromrule=afteraddress]{scrlttr2}
\usepackage[latin1]{inputenc}
\usepackage{graphicx}
\usepackage{amsmath}

\setkomavar{fromname}{Thomas Grubinger\\ Department of Medical Statistics, Informatics and Health Economics \\ Innsbruck Medical University}
\setkomavar{fromaddress}{Sch\"opfstrasse 41/1, 6020 Innsbruck, Austria}
\setkomavar{backaddress}{Dept.~of Med.~Statistics \\ Innsbruck Medical University\\ Sch\"opfstr. 41/1, A-6020 Innsbruck}
\setkomavar{fromphone}{+43 512 9003 70906}
\setkomavar{fromemail}{Thomas.Grubinger@i-med.ac.at}
\setkomavar{signature}{Thomas Grubinger, Achim Zeileis, Karl-Peter Pfeiffer}
\setkomavar{fromlogo}{ \includegraphics*[width=2.8cm]{logo}}
\setkomavar{subject}[]{Revision of  JSS 817}

\begin{document}


\begin{letter}{
  Jan de Leeuw\\
  Department of Statistics\\
  University of California, Los Angeles\\
  United States of America
}

\opening{Dear Prof.~de Leeuw,}

please find attached the revised version of our manuscript
``evtree: Evolutionary Learning of Globally Optimal Classification and Regression Trees in R''
(JSS 817). We would like to thank the reviewer for the careful review and the constructive
comments. We have incorporated all of the reviewer's recommendations, in particular by
cleaning up the notation and providing some more discussion of the illustrations. We
hope to have addressed all concerns in our revision and provide a detailed
point-to-point reply below.

Sincerely,\\[-2cm]
\closing{}

\end{letter}


\newpage


\textbf{\large Point-to-Point Reply}

\bigskip

\textbf{1. Overall Thoughts}

\textit{The paper did a good job of introducing a new tree-based algorithm
overall. A lot of details pertaining to traditional tree models and the concept
of evolutionary algorithms were covered, some of which could be clarified
further. A clean-up in notation and wording would be greatly beneficial. The
code appears easy-to-read and uses parameters consistent with other tree
packages.}

Thanks for pointing these issues out. In fact, providing a unifying but still
simple and easily accessible notation for presenting the framework was one
important objective of the manuscript. While preparing the manuscript, we had
revised the notation a couple of times and apparently did not clean it up
carefully enough. Hence, we appreciate the thorough review and followed
all of its recommendations in this revision. Details of the revision are
provided below.

\medskip

\textit{In the end, the authors gave honest comparisons of their methods to the
traditional rpart and ctree methods, but it is unclear as to when and why evtree
outperforms its competitors apart from the specific chessboard example given in
Section~5.2.}

As the forward-search algorithms just optimze over a subset of conceivable
trees in each step compared to global-search algorithms, the latter are in principle
able to outperform the former when the optimal solution contains more complex
interaction patterns with low main effects. This is now discussed explicitly
in the following new paragraph in Section~5.1:

``From these results, it is not obvious which characteristics drive \texttt{evtree}'s
relative performance. Presumably, for some datasets the forward-search algorithms already
yield trees that are close to optimal, thus leaving little room for further
improvements. In contrast, for other datasets with more complex interaction patterns
(and possibly low main effects) \texttt{evtree}'s global-search strategy is probably
able to provide better predictive accuracy and/or sparser trees.''

\bigskip

\textbf{2. Spelling/Grammatical Errors}

Thanks for catching these. We have corrected all of them and ran again a spell
checker on the manuscript.

\bigskip

\textbf{3. Notation}

\textit{In Section 2, the authors introduce some notation for the tree setup,
but the notation seems inconsistent in the paper. The authors write
$\theta=(v_{n_1}, s_{n_1},\dots,v_{n_{M-1}},s_{n_{M-1}})$, where $n_r \in
\{1,\dots, M_{max}-1\}$ but then also write $v_r$ and $s_r$ where
$r=1,\dots,M-1$. Further, in Section 2.2, the authors talk about $s_r$ and $v_r$
where $r \in \{n_1, \dots, n_{M-1} \}$, which is immediately different from the
indexing given in equation (3). The authors should use a consistent notation for
indexing the $v$ and $s$.}

We cleaned up the notation and a binary tree is now consistently denoted
as $\theta=(v_1, s_1, \dots, v_{M-1}, s_{M-1})$, 
where $v_r \in \{1, \dots ,P\}$ are the splitting \emph{variables} and $s_r$ the
associated \emph{split} rules for the internal \emph{nodes} $r \in \{1,...,M-1\}$.
Thus, in order to keep the notation as simple as possible, the internal nodes of the
tree are just labeled with their number $r \in \{1,...,M-1\}$.


\bigskip

\textbf{4. Case Study}

\textit{The authors have provided very minimal information on the dataset.
(p.~14) The authors write ``Thus, the evolutionary tree uses a different cutoff
in art for book club members that joined in the last year as opposed to older
customers." The text does not explain which variable corresponds to ``join
date", but based on Figure 3, one can indirectly infer that this variable is
first. The authors have described te variable first as the number of months
since the first purchase, but do not say that members join the book club upon
making their first purchase.}

We added a table (Table 2), which describes the variables of the
Bookbinder's Book Club dataset. Also, we avoid the formulation ``join date''
in the text and use ``first purchase'' instead.

\medskip

\textit{The table comparing the misclassification and evaluation function values
for the different methods identify the methods as evtree, rpart, ctree, rpart2,
and ctree2. The paragraph which describes this table instead identifies the
methods as ev, rp, ct, rp2, and ct2. It would be better to make sure the two are
consistent.}

We changed the identifiers in the describing paragraph to also use \texttt{evtree},
\texttt{rpart}, \texttt{ctree}, \texttt{rpart2}, and \texttt{ctree2} instead of the
object names.

\bigskip

\textbf{5. Miscellaneous}

Thanks for pointing all of these out. All of the reviewer's recommendations
have been incorporated into the revised version of the manuscript. The only
point that needs additional explanation is the following:

\medskip

\textit{(p.~10) Third paragraph: Ambiguous ``they" in ``However, as they
represent one of the best..., they are both...". Does ``they" refer to the
parent and offspring trees? Does this mean that despite the $(1+1)$ rule, both
the parent and offspring are both selected to the next generation? This seems
unclear.}

Yes, ``they'' refer to the parent and the offspring. However, the sentence
explains the strong selective pressure of a $(\mu + \lambda)$ strategy with
$\mu>1$ and $\lambda>1$. With a $(1+1)$ strategy, the parent and the offspring
can never be selected for the next round (as it is stated in the last sentence
of the paragraph). 

For better clarification, we replaced ``they'' with ``the parent and the
offspring" and changed the sentence:

``Using a $(\mu + \lambda)$ strategy, a strong selective pressure can occur in
situations as follows."

with:

``Using a $(\mu + \lambda)$ strategy with $\mu>1$ and $\lambda>1$, a strong
selective pressure can occur in situations as follows.".


\end{document}
