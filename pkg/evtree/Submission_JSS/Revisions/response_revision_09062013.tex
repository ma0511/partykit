\documentclass[DIN, pagenumber=false, parskip=half,% 
               fromalign=left, fromphone=true,%  
               fromemail=true, fromurl=false, %
               fromlogo=true, fromrule=false, fromrule=afteraddress]{scrlttr2}
\usepackage[latin1]{inputenc}
\usepackage{graphicx}
\usepackage{amsmath}

\setkomavar{fromname}{Thomas Grubinger}
\setkomavar{fromaddress}{Kampm\"ullerweg 22, 4040 Linz, Austria}
\setkomavar{fromphone}{+43 660 3747597}
\setkomavar{fromemail}{Thomas.Grubinger@utanet.at}
\setkomavar{signature}{Thomas Grubinger, Achim Zeileis, Karl-Peter Pfeiffer}
%\setkomavar{fromlogo}{ \includegraphics*[width=2.8cm]{logo}}
\setkomavar{subject}[]{Revision of  JSS 817}

\begin{document}


\begin{letter}{
  Jan de Leeuw\\
  Department of Statistics\\
  University of California, Los Angeles\\
  United States of America
}

\opening{Dear Prof.~de Leeuw,}

please find attached the revised version of our manuscript
``evtree: Evolutionary Learning of Globally Optimal Classification and Regression Trees in R''
(JSS 817). We would like to thank the reviewer for the careful review and the constructive
comments. We have incorporated all of the reviewer's recommendations,
in particular by including a new section that provides guidance on the parameter selection and
giving a better description of the simulation setup, which led to misunderstandings.
We hope to have addressed all concerns in our revision and provide a detailed
point-to-point reply below.

Sincerely,\\[-2cm]
\closing{}

\end{letter}


\newpage


\textbf{\large Point-to-Point Reply}

\bigskip

\textbf{Originality / Previous research}

\textit{The porting of the an EA to the R package is novel.  There is
little or no novelty in the EA itself, as the authors fairly acknowledge. 
I would like to suggest that they could include the following
reviews or give their readers an up to date survey of the work 
in the field and current research issues.}

Thank you for your suggestion on the two surveys. We concur that these are useful
for the reader and referenced them in the paper.

\bigskip

\textbf{Structure}

\textit{It is in general technically 
correct,  except the description of selection, which is definitely 
not a (1+1) strategy, since that would imply a population size of 1.
It looks to me as if what you actually implement is a steady state 
EA with  deterministic crowding.} 

We agree that the description of our algorithm as (1+1) strategy was not correct.
We fixed the issue according to your recommendations and describe the algorithm now
as a steady state algorithm with deterministic crowding.

\textbf{Structure}

\textit{1 As an introduction to potential users,  it provides little 
guidance on how to set appropriate parameters
such as population size,  crossover probabilities etc.,  and what 
their effect might be on the results they obtain.  It also does not 
talk at all about the major issue with all  stochastic algorithms, 
which is that even on the same data set the results might different 
greatly from run to run. This is confounded by the fact that the 
variability observed will to no small effect depend on the choice 
of selection and variation operators.  There is a chapter all about 
the issue of how to use and compare evolutionary algorithms in 
the Eiben and Smith book the authors cite.} 

We agree that some guidance on the parameter selection can be very useful to the user.
Therefore, we added a new section (6.~Parameter optimization) to the paper, where \textit{evtree} is simulated
with different operator probabilities, number of iterations, and population
sizes. For each setting, \textit{evtree} results on 250~different bootstrap samples are visualized via
boxplots, which should give a good estimate of the underlying variability.


\textit{2.  the ev-tree is a stochastic algorithm,  so the trees evolved 
will vary from run to run,  and will depend on the parameters.  This is not addressed in section of the results where the performance 
of ev-tree is compared to CARt and rtree over 14 data sets and some 
simulated ones.  Thus this whole section suffers from a  severely 
flawed methodology.   There should be multiple runs and some 
indications of the variability in observed performance should be 
provided. There should be far better discussion of why the suite 
of datsets was chosen, what sort of characteristics they might, or 
might not, have.  Ideally appropriate statistical hypothesis tests 
should be used to see whether the observed differences are 
statistically significant.  This could then lead to some ideas of 
the type of problems on which ev-tree does or does not work well.
}

We fully understand your concerns on this point. However, we feel
that our poor description of the simulation setup led to misunderstandings, while
the simulation as such already addressed your concerns.

The results presented in the \textit{Performance  comparison} section are based
on 250~repetition for each algorithm and datasets. Here, we used the common approach of
bootstrapping to build 250 different versions of the original dataset. Following the
approach of Hothorn et~al.\ (2005), the confidence intervals in Figure~4 are adjusted
multiple testing. Thus, confidence intervals that do not encompass the zero-line
indicate statistical significance at the $5\%$ level.

To avoid misunderstandings, we improved the description in the paper as follows:

``The analysis is based on the evaluation of 250~bootstrap
samples for each of the 20~datasets. The misclassification rate on the
\textit{out-of-bag} samples is used as a measure of
predictive accuracy (Hothorn et~al.~2005). Furthermore, the complexity is estimated by the number of
terminal nodes. The results are summarized by the mean differences of the 250 runs -- each corresponding to
one of the 250~different bootstrap samples. For the assessment of significant differences 
in predictive accuracy and complexity, respectively, 
Dunnett's correction from \textit{R}~package \textit{multcomp} (Hothorn, Bretz, and Westfall 2008) was used
for calculating simultaneous $95\%$ confidence intervals on the individual datasets. Confidence intervals
that do not encompass the zero-line indicate significant differences at the $5\%$ level."

Furthermore, we added the following paragraph describing the motivation for the dataset selection:

``The datasets were chosen from different domains and cover a wide range of
dataset characteristics and complexities. The sample sizes of the selected
datasets range from 214 instances (\textit{Glass identification} data) to 19020
instances (\emph{MAGIC gamma telescope}). The number of attributes varies between
4 (\textit{Servo}) and 180 (\emph{DNA}). The types of attributes vary among
datasets, and include datasets which have both categorical and numerical
variables or just one of them. The number of classes for the classification task
vary between 2 and 11 classes."


\textit{3. From the perspective of  a piece of scientific work introducing 
and benchmarking a new  different algorithm, then results should 
ideally be provided for some ``state of the art algorithms" - at 
least for reference.  If there is, as suggested, a relatively 
simpler way of calling weka libraries,  then why is this not 
done?  Why not compare against J48,  and against more modern 
classifiers like SVM etc?}

We included J48 into the simulations showing that similar to rpart and ctree it
is often outperformed with respect to prediction performance. Only for the ``Vowel''
data there is a significant (albeit small) advantage of J48. With respect to complexity,
J48 typically grows much larger trees (also compared to rpart and ctree).

We are aware that there are better methods in terms of predictive accuracy.
However, these methods do not offer the simple
interpretability of classification and regression trees. We added following
paragraph which acknowledges this issue and narrows the scope of the paper.

Classification and regression tree models are widely used and are especially
attractive for many applications, as tree-structured models offer a compact and
intuitive representation. The goal of \code{evtree} is to maintain this
simple tree structure and offer better performance (in terms of predictive
accuracy and/or complexity) than commonly-used recursive partitioning
algorithms. However, in cases where the interpretation of the model is not
important, other ``black-box'' methods including \textit{support vector machines}
(SVM; Vapnik~2005) and tree ensemble methods like \textit{random
forests} (Breiman~2001) typically offer better predictive performance
(Caruana and Niculescu-Mizil~2006;  Hastie, Tibshirani, and Friedman~2009).
 
\end{document}



