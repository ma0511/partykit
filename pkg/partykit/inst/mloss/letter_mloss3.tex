\documentclass[ngerman,uzh]{scrlttr2}
\usepackage[T1]{fontenc}
\usepackage[utf8]{inputenc}

%% ANPASSUNGSMÖGLICHKEITEN
\KOMAoptions{%
%backaddress=false%  Keine Rücksendeadresse im Adressfeld
%,foldmarks=false%  Faltmarken ausschalten
}

\usepackage{babel}


\begin{document}

%% VARIABLEN
% Falls Unterschrift vom definierten Namen abweicht:
\setkomavar{signature}{Torsten Hothorn}
\setkomavar{subject}{Betreff}
%\setkomavar{specialmail}{Interne Post}

\begin{letter}{
Dr.~Cheng Soon Ong \\
MLOSS Editor \\
Journal of Machine Learning Research}

\subject{JMLR-14-133-3 revision}

\opening{Dear Cheng Soon,}

I uploaded a revised version of our manuscript
``\textbf{partykit}: A Modular Toolkit for Recursive Partytioning in R'' to the
Machine Learning Open-Source Software track of the Journal of Machine
Learning Research.

Reviewer~3 suggested to add forest functionality to the \emph{partykit}
package to increase it breadth.
This suggestion was already brought up in the previous revision by
Reviewer~2 and answered to the satisfaction of the first two reviewers:
While the basic idea for setting up a generic class for ``random forests''
is straightforward, many of the details in the implementation are not
obvious at all and deserve much more attention.  
Of course, we could easily copy the code of the old \texttt{party::cforest}
into the new \emph{partykit} package but this would not increase the
usefulness for the practitioner because \emph{party} is easily available. We
therefore decided not to follow the suggestion by reviewer~3 for the same
reasons as explained in the previous round and, in more detail now, the
rebuttal letter.

We revised the manuscript following the Instructions for Authors closely and
will send the copyright forms upon acceptance.

Best regards, \\
\includegraphics[width = .25\textwidth]{TH_Unterschrift}

\end{letter}

\end{document}
