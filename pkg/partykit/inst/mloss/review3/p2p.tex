\documentclass{article}
\usepackage{a4wide}
\usepackage[authoryear,round]{natbib}

\setlength{\parskip}{0.5ex plus0.1ex minus0.1ex}
\setlength{\parindent}{0em}

\title{JMLR-14-133-3: Point-to-Point Reply}
\author{Torsten Hothorn and Achim Zeileis}

\begin{document}

\maketitle

\section*{AE}

\textbf{\textit{%
I am glad to inform you that your paper ``partykit: A Modular Toolkit for
Recursive Partytioning in R'' has been accepted for publication at JMLR Track for Machine Learning
Open Source Software. There are still some minor issues that should be fixed in the final version, which will
be sent out to Reviewer 3 for a quick confirmation.
As pointed out by Reviewer 3, partykit only provides a subset of the functionality of its
earlier cousin party. Porting the old functionality of party to the new
interface of partykit would improve its usefulness.}}

We don't see how moving functionality from one R package to another one
increases the usefulness of the latter for the data analyst. As explained in our response
to referee 2 and also below in our response to the new comments by referee
3, a much improved implementation of \texttt{cforest} and its tools is on our
research agenda but many of the details require further thinking.

\smallskip

\section*{Reviewer 3}

\textbf{\textit{%
The revised version of the paper is reasonable, especially the rewritten
Section 5.  I have no further complaints about the writing of the paper
itself.}}

\textbf{\textit{%
However, I still have issues with the project itself.  My primary concerns
from the previous review are listed below:
\begin{itemize}
 \item The project has little activity.
 \item There appears to be no bug tracker or open mailing list.
 \item The project itself lives in a small niche.
\end{itemize}
The point-by-point rebuttal addresses each of these issues.  The authors
point out that partykit is a mature package and at this point realistically
only requires maintenance.  The authors have also set up a forum (though no
questions have yet been asked).}}

\smallskip

As we already mentioned in our reply to this earlier comment, the use of forums
is not very popular in the R community. In the time after we submitted our
revised version, we received a number of comments and reports by different
users that gave rise to changes in the \emph{partykit} package (svn
revisions 612, 616 and 624). Our experience as maintainers of 15 (TH) and 20
(AZ) CRAN packages shows that this corresponds to ``normal'' activity for
comparable CRAN packages.

\smallskip

\textbf{\textit{%
To address the popularity of partykit, the
authors point out several reverse dependencies of partykit in CRAN, although
they feel that ``such popularity measures [do] not necessarily capture the
quality and scientific value of a software project'' (and this is a statement
I very much agree with).  The authors have pointed out that the algorithms
implemented by partykit are useful.}}

\textbf{\textit{%
I am satisfied with these rebuttals, but the underlying theme of those three
comments can be summarized as ``the project has limited scope''.  My concern
is motivated by criterion 2 of the "Review Criteria: 2. The novelty, breadth, and significance of the contribution (including
evidence of an active user community)''.}}

\textbf{\textit{%
Given that the package is primarily a reimplementation of the authors' own
algorithms "ctree" and "mob" plus a nice, unified API and wrappers for other
recursive partitioning algorithms,}}

\smallskip

In our opinion \emph{partykit} is not a reimplementation of \emph{party}.
A submission describing package \emph{party} was evaluated by the MLOSS
track of the JMLR (as JMLR-07-240) and rejected mainly because the referees
thought (correctly, as we are convinced now) that the design and
implementation of \texttt{ctree}, \texttt{mob} and \texttt{cforest} was not general
enough. Package \emph{partykit} in fact generalises \emph{party} with only one exception
(\texttt{cforest}) for very good reasons as explained below.

\smallskip

\textbf{\textit{%
I am not convinced that the breadth of
this contribution is sufficient, given the much larger breadth of
contributions to JMLR-MLOSS such as Shogun, scikit-learn, Shark, mlpack,
dlib-ml, and other large-scale toolkits.}}

\smallskip

It depends on how one measures ``breadth''. The mentioned toolkits provide a
large set of methods for binary and multiclass classification, regression
and clustering with a focus on SVMs.  \emph{Shark} and \texttt{mlpack} also
contain implementations of tree algorithms for these situations.  Unlike
\emph{partykit}, none of the toolkits provides methods for censored data (survival
analysis) or ordered categorical regression.  With \texttt{mob},
\emph{partykit} allows partitioning of a huge set of models, such as transformation models (Cox,
Weibull), proportional odds models or IRT (item response theory) models
(implemented in package \emph{psychotree} based on \emph{partykit} infrastructure).  

\smallskip

\textbf{\textit{%
Reviewer 2 from the last review cycle comments that the cforest()
functionality from party is not available in partykit, and today this is
still the case.  I would encourage the others to consider expanding the
breadth of partykit to at least entirely encompass the older party package,
and also to use their own clean and nice API for some implementations of
other recursive partitioning algorithms.  This should not be a problem,
given that the clean API allows easy implementation of other algorithms.}}

\smallskip

In our response to Referee 2, we pointed out that a general implementation
of forest-like procedures requires further research into its design. For
example, there are multiple aggregation schemes that need to be taken into
account as a simple majority vote or mean (the original proposal by Breiman)
might be less attractive compared to the understanding of forests as
estimators of conditional distribution functions \citep{Meinshausen_2006}. In
addition, a pure forest is almost useless. Variable selection procedures
(for example based on variable importances and stability selection) and visualisation techniques are
the more interesting (and technically more challenging) things. Simply
transforming currently available \texttt{cforest} functionality from \emph{party} to
\emph{partykit}
will not fix the currently present shortcomings in the design and
implementation of these procedures. In addition, it will not increase the
number of available procedures because the \emph{party} package is still maintained
on CRAN and will be available for the forseeable future.

\smallskip

\textbf{\textit{%
Increasing the breadth of the package should also increase its visibility,
usage, and popularity.
Overall, I do not want to give the authors the idea that I think their
package is not a good example of well-thought-out software engineering. 
partykit is well-designed and extensively documented.  But my primary
concern is a lack of breadth.
}}

\smallskip

We are confident that the current \emph{partykit} infrastructure and API
will motivate other researcher to implement novel tree-based procedures that
in turn will increase the breath of recursive partitioning methods available
to R users.  For example, package \emph{vcrpart}
(\texttt{http://CRAN.R-project.org/package=vcrpart}) is a newly released
package that implements varying coefficient models based on \emph{partykit}
infrastructure.  The package was developled completely independent of the
\emph{partykit} authors and, btw, the authors of \emph{vcrpart} never posted
a question by email or in the forum which we tend to attribute to the
extensive package documentation of \emph{partykit}.  Novel procedures such
as the one implemented in \emph{vcrpart} is exactly the notion of ``breath''
we had in mind when designing \emph{partykit}.  We expect other research
groups to reuse \emph{partykit infrastructure} in the future, especially
after the package and corresponding description was made popular by MLOSS.

\bibliographystyle{plainnat}
\bibliography{../ref.bib}

\end{document}
