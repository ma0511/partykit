\documentclass{article}
\usepackage{a4wide}
\usepackage[authoryear,round]{natbib}

\setlength{\parskip}{0.5ex plus0.1ex minus0.1ex}
\setlength{\parindent}{0em}

\title{JMLR-14-133-3: Point-to-Point Reply}
\author{Torsten Hothorn and Achim Zeileis}

\begin{document}

\maketitle

\section*{AE}

\textbf{\textit{%
I am glad to inform you that your paper ``partykit: A Modular Toolkit for
Recursive Partytioning in R'' has been accepted for publication at JMLR Track for Machine Learning
Open Source Software. There are still some minor issues that should be fixed in the final version, which will
be sent out to Reviewer 3 for a quick confirmation.
As pointed out by Reviewer 3, partykit only provides a subset of the functionality of its
earlier cousin party. Porting the old functionality of party to the new
interface of partykit would improve its usefulness.}}

\textbf{\textit{%
After these changes have been completed, I would like to ask you to prepare
your paper for final submission following the instructions given at
http://www.jmlr.org/format/authors-guide.html.
Please fill out both the permission to publish form and the source code
release form.}}

\smallskip

\section*{Reviewer 3}

\textbf{\textit{%
The revised version of the paper is reasonable, especially the rewritten
Section 5.  I have no further complaints about the writing of the paper
itself.}}

\textbf{\textit{%
However, I still have issues with the project itself.  My primary concerns
from the previous review are listed below:
\begin{itemize}
 \item The project has little activity.
 \item There appears to be no bug tracker or open mailing list.
 \item The project itself lives in a small niche.
\end{itemize}
The point-by-point rebuttal addresses each of these issues.  The authors
point out that partykit is a mature package and at this point realistically
only requires maintenance.  The authors have also set up a forum (though no
questions have yet been asked).  To address the popularity of partykit, the
authors point out several reverse dependencies of partykit in CRAN, although
they feel that ``such popularity measures [do] not necessarily capture the
quality and scientific value of a software project'' (and this is a statement
I very much agree with).  The authors have pointed out that the algorithms
implemented by partykit are useful.}}

\textbf{\textit{%
I am satisfied with these rebuttals, but the underlying theme of those three
comments can be summarized as ``the project has limited scope''.  My concern
is motivated by criterion 2 of the "Review Criteria: 2. The novelty, breadth, and significance of the contribution (including
evidence of an active user community)''.}}

\textbf{\textit{%
Given that the package is primarily a reimplementation of the authors' own
algorithms "ctree" and "mob" plus a nice, unified API and wrappers for other
recursive partitioning algorithms, I am not convinced that the breadth of
this contribution is sufficient, given the much larger breadth of
contributions to JMLR-MLOSS such as Shogun, scikit-learn, Shark, mlpack,
dlib-ml, and other large-scale toolkits.}}


\textbf{\textit{%
Reviewer 2 from the last review cycle comments that the cforest()
functionality from party is not available in partykit, and today this is
still the case.  I would encourage the others to consider expanding the
breadth of partykit to at least entirely encompass the older party package,
and also to use their own clean and nice API for some implementations of
other recursive partitioning algorithms.  This should not be a problem,
given that the clean API allows easy implementation of other algorithms.
Increasing the breadth of the package should also increase its visibility,
usage, and popularity.
Overall, I do not want to give the authors the idea that I think their
package is not a good example of well-thought-out software engineering. 
partykit is well-designed and extensively documented.  But my primary
concern is a lack of breadth.
}}

\end{document}
