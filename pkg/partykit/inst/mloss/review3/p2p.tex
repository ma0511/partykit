\documentclass{article}
\usepackage{a4wide}
\usepackage[authoryear,round]{natbib}

\setlength{\parskip}{0.5ex plus0.1ex minus0.1ex}
\setlength{\parindent}{0em}

\title{JMLR-14-133-3: Point-to-Point Reply}
\author{Torsten Hothorn and Achim Zeileis}

\begin{document}

\maketitle

\section*{AE}

\textbf{\textit{%
I am glad to inform you that your paper ``partykit: A Modular Toolkit for
Recursive Partytioning in R'' has been accepted for publication at JMLR Track for Machine Learning
Open Source Software.
}}

\smallskip

Thank you for the good news, this is much appreciated.

\smallskip

\textbf{\textit{%
There are still some minor issues that should be fixed in the final version, which will
be sent out to Reviewer~3 for a quick confirmation.
As pointed out by Reviewer~3, partykit only provides a subset of the functionality of its
earlier cousin party. Porting the old functionality of party to the new
interface of partykit would improve its usefulness.}}

\smallskip

This suggestion was already brought up in the previous revision by Reviewer~2
and answered to the satisfaction of the first two reviewers: While the basic idea for
setting up a generic class for ``random forests'' is straightforward, many of the
details in the implementation are not obvious at all and deserve much more attention.
See also our full response included below.

Of course, we can easily copy the code of the old \texttt{party::cforest} into the
new \emph{partykit} package. However, the usefulness for the practitioner would not
really be increased by this (given that \emph{party} is easily available). More
importantly, if we take the recommendations of Reviewer~2 seriously (which we do),
then we should not just copy existing code but build new extensible infrastructure
for forests as well. If we only copy the old code now, future developments are likely
to change at least parts of the interface and thus potentially break full
backward compatibility. Therefore, we would much rather follow Reviewer~2 and devote
more future work to this important topic.

The full question and response from the previous round of reviews was:
%
\begin{quote}
\textbf{\textit{%
How well-adapted are the tools in partykit for implementing random forest
or other ensemble type approaches? I note that partykit does not (yet?) have
an equivalent of the cforest() function that is in the older party package.
}}

In principle, it would be straightforward to extend the idea of the \texttt{party}
class. This is stored as the data information (including terms etc.) and a separate recursive
\texttt{partynode} representing the tree. Hence, the natural class for forests
(possibly named \texttt{parties} or \texttt{orgy}, maybe) would be the data
information plus a list of \texttt{partynode}s. This exploits that only the tree
structures vary across the ensemble but not the data. However, the class itself
would only be the beginning and one would additionally need a unified way to produce
meaningful plots (e.g., variable importances, partial dependencies), predictions,
etc. These are less straightforward and need further time and thinking prior
to implementation. This is on our to-do list, though.
\end{quote}

\smallskip

\section*{Reviewer 3}

\textbf{\textit{%
The revised version of the paper is reasonable, especially the rewritten
Section 5.  I have no further complaints about the writing of the paper
itself.}}

\smallskip

Thank you, this is much appreciated.

\smallskip

\textbf{\textit{%
However, I still have issues with the project itself.  My primary concerns
from the previous review are listed below:
\begin{itemize}
 \item The project has little activity.
 \item There appears to be no bug tracker or open mailing list.
 \item The project itself lives in a small niche.
\end{itemize}
The point-by-point rebuttal addresses each of these issues.  The authors
point out that partykit is a mature package and at this point realistically
only requires maintenance.  The authors have also set up a forum (though no
questions have yet been asked).}}

\smallskip

The lack of questions in the forum is due to the fact that the forum was not yet
advertised. However, it will be once the manuscript is published in the MLOSS
track and we expect that the forum will be used more actively then.

So far, we (like many other package maintainers in the \textsf{R} community)
have simply used e-mails for correspondence about problems, bugs, feature requests, etc.
For example, in the one month since resubmission of the revised manuscript to the
MLOSS track we received a number of comments and reports by different
users that gave rise to several small changes in the \emph{partykit} package (SVN
revisions 612, 616, and 624).

\smallskip

\textbf{\textit{%
To address the popularity of partykit, the
authors point out several reverse dependencies of partykit in CRAN, although
they feel that ``such popularity measures [do] not necessarily capture the
quality and scientific value of a software project'' (and this is a statement
I very much agree with).  The authors have pointed out that the algorithms
implemented by partykit are useful.}}

\textbf{\textit{%
I am satisfied with these rebuttals, but the underlying theme of those three
comments can be summarized as ``the project has limited scope''.  My concern
is motivated by criterion 2 of the "Review Criteria: 2. The novelty, breadth,
and significance of the contribution (including evidence of an active user
community)''.}}

\textbf{\textit{%
Given that the package is primarily a reimplementation of the authors' own
algorithms \texttt{ctree} and \texttt{mob} plus a nice, unified API and wrappers for other
recursive partitioning algorithms,}}

\smallskip

This appears to convey that the reimplementation of \texttt{ctree} and \texttt{mob}
is not very substantial and only the new unified API is an original contribution.
However, this is not correct.

While the underlying conceptual algorithms were already published in two papers
in the \emph{Journal of Computational and Graphical Statistics}, the software
implementation was not yet published in a journal. Originally, we submitted the
\emph{party} package to the MLOSS track of the JMLR as JMLR-07-240 which was rejected
mainly because the reviewers felt (correctly, as we are convinced now) that the design and
implementation was not general enough. This prompted a reimplementation in \emph{partykit},
essentially from scratch, which was then resubmitted as JMLR-14-133.

Therefore, the entire package along with its extensive documentation should be
seen as an original contribution. It supersedes the \emph{party} implementation
in all aspects with only one notable exception (the \texttt{cforest} function,
for very good reasons as explained below).

\smallskip

\textbf{\textit{%
I am not convinced that the breadth of this contribution is sufficient, given the much larger breadth of
contributions to JMLR-MLOSS such as Shogun, scikit-learn, Shark, mlpack, dlib-ml, and other large-scale toolkits.
}}

\smallskip

This depends on the dimensions in which ``breadth'' is assessed. The above-mentioned
toolkits provide a large set of methods for binary and multiclass classification, regression
and clustering with a focus on SVMs. Some of the packages above, e.g., \emph{Shark} and \emph{mlpack}, also
contain implementations of tree algorithms for these situations.  Unlike
\emph{partykit}, none of the toolkits provides dedicated methods for targets
with structures other than binary/multiclass/numeric. For example: censored data (including survival analysis), ordered
categorical regression (e.g., proportional odds models), item response theory (IRT),
paired comparisons, multivariate data with mixed types (e.g., several endpoints in a clinical study
with mixed numeric/categorical/censored measurement scales). Through the use
of nonparametric transformations (\texttt{ctree}) or parametric statistical models (\texttt{mob})
all of the cases above can be addressed in \emph{partykit}.

For example, this has enabled ourselves to successfully develop dedicated psychometric
tree models in package \emph{psychotree} for IRT and paired comparison data -- but it also
has enabled the authors of the \emph{quint} package to establish trees for detecting
treatment-subgroup interactions in clinical data. For a third very recent example
(package \emph{vcrpart}) see the comments below.


\smallskip

\textbf{\textit{%
Reviewer 2 from the last review cycle comments that the \texttt{cforest()}
functionality from party is not available in partykit, and today this is
still the case.  I would encourage the others to consider expanding the
breadth of partykit to at least entirely encompass the older party package,
and also to use their own clean and nice API for some implementations of
other recursive partitioning algorithms.  This should not be a problem,
given that the clean API allows easy implementation of other algorithms.
}}

\smallskip

As pointed out in our response to Reviewer~2, the mere representation of the
forest (= list of trees plus data and metainformation) is straightforward.
The real challenge is to provide a \emph{unified and extensible} collection of methods
for computations on these forests (as available for single trees): e.g.,
aggregation of predictions across trees, assessement of variable importance,
visualization of partial dependence. While Breiman's (and Cutler's) original
proposal (and software) tackled these problems to some degree, their solutions
have been either shown to perform poorly in certain situations or are not
general enough in others.

For example, the simple majority vote or mean aggregation
might be less attractive compared to the understanding of forests as
estimators of conditional distribution functions \citep{Meinshausen:2006},
especially when applying them to more complex data (censored, ordered, IRT, \dots).
Or as another example, the originally proposed variable importance measures are
artificially biased when applied to predictors with different scales
\citep{Strobl+Boulesteix+Kneib:2008} compromising their suitability in applied
work (e.g., in bioinformatics).

Of course, we can simply move the existing \texttt{cforest} code from \emph{party}
to \emph{partykit} but this would not solve the problems discussed above. These
deserve further thought and work, both conceptually and computationally.

\smallskip

\textbf{\textit{%
Increasing the breadth of the package should also increase its visibility, usage, and popularity.
Overall, I do not want to give the authors the idea that I think their
package is not a good example of well-thought-out software engineering. 
partykit is well-designed and extensively documented.  But my primary
concern is a lack of breadth.
}}

\smallskip

We are confident that the current \emph{partykit} infrastructure and API
will motivate other researchers to implement novel tree-based procedures that
in turn will increase the breadth of recursive partitioning methods available
in \textsf{R}. 

One example for such a ``success story'' that emerged while our revised manuscript was under review
is the \emph{vcrpart} package (published on \texttt{http://CRAN.R-project.org/package=vcrpart}
in September).
This is a completely new package implementing varying coefficient trees for
ordinal 2-stage linear mixed models based on \emph{partykit} infrastructure.
By the way, apart from a few small questions about implementation
details, the \emph{vcrpart} authors apparently did not need to be in contact with us and have
been able to leverage the \emph{partykit} infrastructure based on its documentation.
Moreover, the main author of \emph{vcrpart} submitted a few bug reports
about smaller glitches in \emph{partykit} by e-mail (not the forum) that we have
subsequently fixed.

Novel procedures such as the one implemented in \emph{vcrpart} are exactly the
notion of ``breadth'' we had in mind when designing \emph{partykit} and that
the package aims to facilitate. We expect other research groups to pick up the
\emph{partykit} infrastructure in the future, especially if it is given increased
visibility in the MLOSS track of JMLR.

\bibliographystyle{plainnat}
\bibliography{../ref.bib}

\end{document}
