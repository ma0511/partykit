\documentclass[ngerman,uzh]{scrlttr2}
\usepackage[T1]{fontenc}
\usepackage[utf8]{inputenc}

%% ANPASSUNGSMÖGLICHKEITEN
\KOMAoptions{%
%backaddress=false%  Keine Rücksendeadresse im Adressfeld
%,foldmarks=false%  Faltmarken ausschalten
}

\usepackage{babel}


\begin{document}

%% VARIABLEN
% Falls Unterschrift vom definierten Namen abweicht:
\setkomavar{signature}{Torsten Hothorn}
\setkomavar{subject}{Betreff}
%\setkomavar{specialmail}{Interne Post}

\begin{letter}{
Drs.~Murphy and Schölkopf \\
Editors-in-chief \\
Journal of Machine Learning Research}

\subject{JMLR-14-133 resubmission}

\opening{Dear Drs.~Murphy and Schölkopf,}

my co-author Achim Zeileis and I would like to resubmit the manuscript
``\textbf{partykit}: An Open-Source Toolkit for Recursive Partytioning in R'' to the
Machine Learning Open-Source Software track of the Journal of Machine
Learning Research.

We would like to thank the referees for their constructive reviews and
action editor Cheng Soon Ong for overseeing the review process. While
preparing this resubmission, we took all issues discussed by the referees
into account and revised both the \emph{partykit} package and the four-page
description. A detailed description of our revision can be found in the
point to point reply.

The package web page is \texttt{http://CRAN.R-project.org/package=partykit}
and points to the revised version 0.8-1 distributed under the GPL (2 or 3).
This version is also uploaded for review. 

Sincerely yours, \\
\includegraphics[width = .25\textwidth]{TH_Unterschrift}

\end{letter}

\end{document}