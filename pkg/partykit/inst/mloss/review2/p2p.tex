\documentclass{article}
\usepackage{a4wide}
\usepackage[authoryear,round]{natbib}

\setlength{\parskip}{0.5ex plus0.1ex minus0.1ex}
\setlength{\parindent}{0em}

\title{Revision of JMLR-14-133: Point-to-Point Reply}
\author{Torsten Hothorn and Achim Zeileis}

\begin{document}

\maketitle

\section*{AE}

\textbf{\textit{%
I regret to inform you that your submission ``partykit: A modular toolkit
for recursive partytioning in R'' has been rejected from publication at JMLR
Track for Machine Learning Open Source Software.  Two of the reviewers of
the previous version were enlisted for this version, and one new reviewer
was added.  While the reviewers voted for ``accept'', the actual reviews
were less positive and raised several serious concerns about the project. 
If the authors choose to resubmit the project, please include a description
addressing the reviewers' concerns.
}}

\smallskip

Thank you for the constructive reviews and feedback and for giving us the
chance to revise our manuscript/software and submit it to the MLOSS track
again. Please find attached the revision of our four-page manuscript
and a new version (0.8-1) of the \emph{partykit} package which is already
available from \texttt{http://CRAN.R-project.org/package=partykit} as well.

The most important changes are a completely rewritten Section~5 in the four-page
manuscript (Discussion and Outlook) and an extended Section~3 in the \texttt{constparty}
vignette (showing how new tree algorithms can be implemented using \emph{partykit}).
Moreover, an online support forum for \emph{partykit} has been established
on \textsf{R}-Forge at \texttt{https://R-Forge.R-project.org/forum/forum.php?forum\_id=852}.
Further details of the revision are discussed in the point-to-point reply
to the reviewers.

Note that the revised four-page manuscript now has the acknowledgments (four lines)
on the fifth page along with the references. We are not sure whether this
conforms with the MLOSS style guidelines: if not, we will, of course, streamline
the main manuscript to place the acknowledgments on the first four pages as well.

We hope that the \emph{partykit} package and the four-page description are
suitable for consideration in the MLOSS track of JMLR now.

\section*{Reviewer 1}

\textbf{\textit{%
This is a clean and concise description of an interesting set of tools for
data analysis with recursive partitioning.  It is mainly a description or
announcement, and has minimal polemic or pedagogical value.
}}

\smallskip

Yes, this follows advice that we were given in the rejection of an earlier
submission of our \emph{party} package (JMLR-07-240). One of the reviewers
requested that the four-page manuscript should focus on broadbrush
information and guide the reader to more detailed documentation available
along with the software:
\begin{quote} 
\emph{Surely it is possible to give broadbrush information. There should be direct
reference to web-based materials that give more detailed information and examples.}
\end{quote}

\medskip

\textbf{\textit{%
It will be a useful reference for those a basic outline of the software's
capabilities, and provides a useful bibliography.
}}

Thank you.

\section*{Reviewer 2}

\textbf{\textit{%
This paper describes work that, in my understanding, does two things:
\begin{enumerate}
\item It provides infrastructure that can in principle be used with any
      recursive partitioning implementation, and that at this point 
      interfaces to several different widely used implementations, 
      additional to those directly available in R's partykit package.
\item It provides implementations of two specialized tree-based methods.
\end{enumerate}
}}

\smallskip

Thank you for this nice and concise summary of the software and accompanying
four-page manuscript.

When we started working on \emph{partykit}, we redesigned all classes from the
earlier \emph{party} package from scratch and also reimplemented \texttt{ctree} and \texttt{mob}
with relatively little overlap between the old and new code bases. As a
consequence of the much cleaner \emph{partykit} design, the code base for both
\texttt{ctree} and \texttt{mob} could be reduced considerably while at the same time
both implementations gained more flexibility and additional options by leveraging
the new class structure.

\medskip

\textbf{\textit{%
The software has a useful role in providing commonality across diverse
recursive partitioning methods and implementations: for structuring printed
output, for generating predictions, and for plotting results.  Very
extensive documentation, some distributed with the package, is available
from references given in the paper.
}}

\smallskip

Thank you.

\medskip

\textbf{\textit{%
Quibbles and queries are:
}}
\begin{enumerate}
\item 
\textbf{\textit{%
The legend for Table 1 begins: ``Selected tree algorithms than can be
interfaced through partykit...''  Surely it is the output from selected
implementations that can be interfaced, not the algorithm used.
}}

\smallskip

Yes, thank you. We completely agree with you on the distinction between
`algorithm' and its `implementation' and improved the caption accordingly.

\medskip

\textbf{\textit{%
The word `algorithm' seems to me anyway inappropriate in this context. The
same (or very similar) algorithm may have different implementations.  Different
methods with their inevitably different algorithms (including, in partykit:
classical `constant' trees, model-based trees, and conditional inference
trees) further widen the range of implementations.
}}

\smallskip

We completely agree and try to use ``algorithm'' for the conceptual method
``implementation'' for an accompanying software throughout the paper now.

\item 
\textbf{\textit{%
The distinction between implementation (what the authors call `algorithm')
and output is crucial also because as far as I am able to judge, partykit's
abilities are limited to taking over and processing some part only of the
output from, e.g., R's rpart package.  As I understand, there is no ability
to use or process, e.g., results from rpart's on the fly cross-validation.
}}

\smallskip

Yes, this is correct. It is not possible to directly work on the \texttt{rpart}
results after coercion to the \texttt{constparty} class.  The reason is that not
all constant-fit trees are based on a risk or loss function and hence there
is no uniform approach for performing cost-complexity pruning.
Therefore, the current implementation assumes that the pruning has already been
performed within \texttt{rpart} (or \texttt{J48}) prior to coercion.

\medskip

\textbf{\textit{%
Would it be straightforward to incorporate structure that is able to
accommodate such information?  See also point 3 that now follows.
}}

\smallskip

Yes, one `just' needs to define a new class that inherits from \texttt{constparty},
possibly containing a cost-complexity table in the \texttt{info} slot with
appropriate methods for determining the nodes to be pruned and then call
the \texttt{nodeprune} function. An example for the use of \texttt{nodeprune} is
given in Section~3 of the \texttt{constparty} vignette. Furthermore,
this type of approach (based on information criteria rather than cost-complexity)
has also been coded into the \texttt{mob} function (while not having been available
in the old \emph{party} implementation).

\item 
\textbf{\textit{%
I note the comment on the help page for the partykit function ctree() that
the inference process is based on permutation tests, with one or other type
of adjustment for multiple testing.  Then follows the strong claim that:
This statistical approach ensures that the right sized tree is grown and no
form of pruning or cross-validation or whatsoever is needed.  I think it
pertinent to ask what has been done to judge the effectiveness of this
approach, assessed using separate test data, against the approach used in,
e.g., rpart or CART?  This is an issue for the help page, rather than
perhaps for the paper.
}}

\smallskip
As discussed in Section~3.4 of \cite{Hothorn+Hornik+Zeileis:2006}, two
different views on this issue can be taken. Either a nominal level for
the statistical tests can be selected that is useful for the sample size
at hand or the significance level can be treated as a hyperparameter that
can be tuned when learning the tree. The manual page has been changed
to better reflect this.

For small to medium-sized data sets, conditional inference trees with
nominal level of 5\% (and appropriate adjustment for multiplicity) were
shown to produce trees of the correct size \citep[simulation results
in][]{Hothorn+Hornik+Zeileis:2006}.

\item 
\textbf{\textit{%
How well-adapted are the tools in partykit for implementing random forest
or other ensemble type approaches? I note that partykit does not (yet?) have
an equivalent of the cforest() function that is in the older party package.
}}

\smallskip

In principle, it would be straightforward to extend the idea of the \texttt{party}
class. This is stored as the data information (including terms etc.) and a separate recursive
\texttt{partynode} representing the tree. Hence, the natural class for forests
(possibly named \texttt{parties} or \texttt{orgy}, maybe) would be the data
information plus a list of \texttt{partynode}s. This exploits that only the tree
structures vary across the ensemble but not the data. However, the class itself
would only be the beginning and one would additionally need a unified way to produce
meaningful plots (e.g., variable importances, partial dependencies), predictions,
etc. These are less straightforward and need further time and thinking prior
to implementation. This is on our to-do list, though.

\end{enumerate}

\textbf{\textit{%
At the very least, I would like to see brief comments on where splitting and
accuracy assessment strategies, short of use of separate test data, fit in
or around the toolkit.
}}

\smallskip

The functionality for learning the tree and assessing its accuracy has to
be provided by developers of new tree methods and is not readily provided
by \emph{partykit} itself (although certain building blocks are available).
This is pointed out more clearly in the rewritten discussion at the end of
the manuscript.

In our experience, this separation of the general infrastructure makes sense
because the details of how variables and splits are selected and how trees are
pruned differ considerably between algorithms. Our previous implementations
in the \emph{party} package made the mistake to blend rather general and very
specific features into the same tree classes. And then when we wanted to implement or
interface a different tree model, we either needed to abuse some slots of existing
classes or had to start writing a new class.

The \texttt{constparty} vignette contains a worked example (Section 3) of
implementing a non-trivial tree algorithm from scratch with only three
relatively short \textsf{R} functions. This should highlight that the
developer can concentrate on the technical details of learning
the tree while using the \emph{partykit} infrastructure for representing and
further processing the resulting tree. In the latest release of \emph{partykit}
we enhanced this section in the vignette by illustrating how model
assessment strategies could be implemented for such a tree.

\medskip

\textbf{\textit{%
To what extent are toolkit functions independent of
the methodology used for the partykit functions ctree() and mob()?
}}

\smallskip

All the base classes (\texttt{party}, \texttt{partynode}, \texttt{partysplit}) are
completely independent of \texttt{ctree} and \texttt{mob}. The classes
\texttt{constparty} and \texttt{modelparty} are also agnostic and can represent
trees learned by other software (as clearly demonstated for \texttt{constparty})
but, of course, have been made flexible enough to accomodate all features
of \texttt{ctree} and \texttt{mob}, respectively.

\medskip

\textbf{\textit{%
If necessary, some of the historical detail in the overview might be replaced
by references to discussion that appears elsewhere, in order to make way for
comment on these points.
}}

\smallskip

We have extended the discussion and outlook in the four-page manuscript. Provided
that we may keep the acknowledgments along with the references on the fifth page,
no further shortening was necessary. However, if this does not conform with the
MLOSS requirements we will follow your advice and streamline the introduction to
save some space.

\section*{Reviewer 3}

\textbf{\textit{%
The paper describes an R library, `partykit', that appears to be a unified
way of representing decision tree structures (or recursive partitioning
structures) in R and manipulating or visualizing them.  The authors note
that their library contains support for representing the output of trees
created with other packages, and the paper appears to function as a very
basic tutorial with links to very detailed R vignettes.  The package also
implements a few of the authors' own recursive partitioning algorithms.
}}

\smallskip

As already pointed out in our reply to Reviewer~1, this format of the
four-page manuscript pointing to the vignettes and other documentation was
requested by one of the reviewers of an earlier version of the submission.

\medskip

\textbf{\textit{%
The project seems to have fairly minor activity, and appears to have had a
Google Summer of Code project in previous years.  The subversion repository
seems to indicate that the code is still being (somewhat) actively developed
despite a relative lack of commits between 2009 and 2013.
}}

\smallskip

The package is developed by only two researchers, hence there can sometimes
be longer breaks in the commits, e.g., due to working on other projects,
applying for professor positions and switching jobs, or starting families.

However, during 2009 and 2013 the package was actively maintained with six
CRAN releases, e.g., fixing bugs or incorporating smaller user requests. Also
during this time, two offspring packages (\emph{evtree} and \emph{psychotree}) were
developed that leverage the \emph{partykit} infrastructure. These brought
only some smaller design or implementation deficits to light that did not
necessitate (re-)writing/modifying large parts of the \emph{partykit} code.

Therefore, we feel that \emph{partykit} has matured by now and pure activity
is not necessarily an indicator of the quality of the code.

\medskip

\textbf{\textit{%
There appears to be no bug tracker or open mailing list.
}}

\smallskip

E-mails to the maintainer address were always possible and this is a rather common
way of communicating with CRAN package maintainers (and developers). In the past,
this means of communication with us was used fairly regularly by both end-users
and other package developers.

As \textsf{R}-Forge easily allows for having mailing lists, forums, etc., we
have added a discussion forum (\texttt{https://R-Forge.R-project.org/forum/forum.php?forum\_id=852})
and also point the readers to this in the rewritten Section~5 of our manuscript.

\medskip

\textbf{\textit{%
There is knowledge of the package in the R ecosystem, but partykit does not seem
to be as popular as other R packages (admittedly, partykit is a niche product).
}}

\smallskip

There are certainly more popular (but typically also more generally applicable) packages
in the \textsf{R} world. However, we feel that \emph{partykit} is doing quite
well. 

As of now, the CRAN package web page lists seven reverse dependencies (in two of
which we have no involvement at all and two others that are not maintained by us).
Only 6\% of all CRAN packages have more reverse dependencies than that. And we hope
that some of the 24~reverse dependencies of \emph{party} will switch to \emph{partykit}
(only 2\% of all CRAN packages have more than 24 reverse dependencies) now that the
recommended reference implementations of \texttt{ctree} and \texttt{mob} are in \emph{partykit}.
Moreover, according to Google Scholar the \texttt{ctree} and \texttt{mob} papers
published in \emph{Journal of Computational and Graphical Statistics} were cited
643 and 103 times, respectively.

Having said that, we nevertheless feel that such popularity measures not necessarily
capture the quality and scientific value of a software project.

\medskip

\textbf{\textit{%
Support for trees produced by other libraries is provided, but I am not
certain how appealing the infrastructure will be to developers who want to
represent trees from other systems.  It seems like for non-PMML formatted
trees, implementing a conversion layer to partykit objects wouldn't be any
easier or harder than other packages. Clearly, for developing new tree
algorithms, though, partykit would be a good choice.
}}

\smallskip

Our experience is not that any tree model can be converted in the same easy
or hard way to any other package.

While implementing the original code for \texttt{ctree} 10~years ago, we could
not easily leverage infrastructure that existed at the time because it had way too many
algorithm-specific things hardwired into it. While \texttt{rpart}
always allowed to change risk functions, doing something outside of the
CART-way of finding splits was not really possible. Also printing and plotting
facilities were somewhat limited and not able to directly include $p$-values
in the inner nodes, for example.
Our \emph{party} implementation of \texttt{ctree} then made the same mistake of
wiring too many algorithm-specific requirements into the class. Hence, it really
had to be somewhat abused to capture \texttt{mob} trees. And the multi-way splits of
\texttt{J48} from \emph{RWeka} could not be captured by either \emph{rpart}
or \emph{party}.

These experiences led to the \emph{partykit} package. The reimplementation of
\texttt{ctree}, \texttt{mob} and the new implementations of trees in \emph{evtree} and \emph{psychotree}
turned out to be rather straightforward and we personally did benefit a lot
from the \emph{partykit} infrastructure already. We suppose this is also true for
the developers of the \emph{NHEMOtree} and \emph{quint} packages.

\medskip

\textbf{\textit{%
Section 5 is very qualitative and vague.  The authors should definitely
improve this in future revisions.  It is written, ``lean data structures are
employed", ``the `party' structures are very lean", and also ``The recursive
node structure itself is extremely lightweight" but without any sort of
quantitative details, it's difficult to believe those statements without
diving into the code.
}}

\smallskip

Yes, your are right, thank you for pointing this out. We have completely
rewritten Section~5, refraining from going into details about memory and
speed comparisons and highlighting the flexibility/extensibility of the
package instead.

\medskip

\textbf{\textit{%
In my opinion the authors could do one of these things to validate their
claims: provide proof that the data structures are minimal (with respect to
other implementations), show that very large tree structures can be easily
represented and handled by partykit, or simply leave out or significantly
shorten the discussion of speed and memory usage entirely.
}}

\smallskip

We have added a demo to the package that briefly illustrates the
memory and speed considerations that we alluded to in the initial submission
by comparison to \texttt{rpart} and \texttt{J48}.
This can be run interactively in \textsf{R} via
\texttt{demo("memory-speed", package = "partykit")}. However, it is not
mentioned in the manuscript as this is not the main focus of the package.

\medskip

\textbf{\textit{%
Overall, partykit is a well-designed and extensively documented package, but
I have some reservations about the project's activity level and popularity
going into the future.  The developers should consider setting up a mailing
list or bug tracker to make the project more open to new contributors.
}}

\smallskip

Based on our experience with the \emph{party} package for 10 years now, the fact
that other developers started packages building on \emph{partykit}, and the feedback via
e-mail and at conferences (most importantly the recent ``Workshop on Classification
and Regression Trees'' at NUS in March 2014), we feel sufficiently confident
about the popularity of \emph{partykit}.

As for the openness to new contributors: Both end-users and package developers
are invited to contact us personally, via e-mail or through the newly established
forum. If (reasonable) changes in \emph{partykit} are necessary we try to accomodate them.
More typically, however, new contributions based on \emph{partykit} can be more easily
wrapped into new packages which may (as in the case of \emph{evtree}) or may
not (as for \emph{quint} and \emph{NHEMOtree}) lead to co-development with us.

\bibliographystyle{plainnat}
\bibliography{../ref.bib}


\end{document}
