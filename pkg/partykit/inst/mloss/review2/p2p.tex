
\documentclass{article}
\usepackage{natbib}

\title{Rebuttal JMLR 14-133}
\author{Torsten Hothorn and Achim Zeileis}

\begin{document}

\maketitle

\section*{AE}

\textit{
I regret to inform you that your submission ``partykit: A modular toolkit
for recursive partytioning in R'' has been rejected from publication at JMLR
Track for Machine Learning Open Source Software.  Two of the reviewers of
the previous version were enlisted for this version, and one new reviewer
was added.  While the reviewers voted for ``accept'', the actual reviews
were less positive and raised several serious concerns about the project. 
If the authors choose to resubmit the project, please include a description
addressing the reviewers' concerns.}

\section*{Referee 1}

\textit{
This is a clean and concise description of an interesting set of tools for
data analysis with recursive partitioning.  It is mainly a description or
announcement, and has minimal polemic or pedagogical value.}

When preparing this revision of our earlier version JMLR-07-240, we closely
followed the suggestions given in the corresponding reviews. Our
understanding of this remark of Referee 3 on JMLR-07-240
\begin{quote} 
Surely it is possible to give broadbrush information. There should be direct
reference to web-based
materials that give more detailed information and examples.
\end{quote}
was that the four page description should really focus on 
broadbrush informtion, with all the interesting details being explained and
discussed in the packages's documentation.

\textit{It will be a
useful reference for those a basic outline of the software's capabilities,
and provides a useful bibliography.}

Thank you.

\section*{Referee 2}

\textit{
This paper describes work that, in my understanding, does two things:
\begin{enumerate}
\item It provides infrastructure that can in principle be used with any
      recursive partitioning implementation, and that at this point 
      interfaces to several different widely used implementations, 
      additional to those directly available in R's partykit package.
\item It provides implementations of two specialized tree-based methods.
\end{enumerate}
}

Thank you for this nice and concise summary of the four page description.

When we started working on partykit, we redesigned all classes from the
earlier party package from scretch and also reimplemented ctree and mob
without much overlap between the party and partykit code bases. As a
consequence of the much cleaner partykit design, the code base for both
ctree and mob could be reduced considerably while at the same time making
both implementations more powerful due to more flexible options.

\textit{
The software has a useful role in providing commonality across diverse
recursive partitioning methods and implementations: for structuring printed
output, for generating predictions, and for plotting results.  Very
extensive documentation, some distributed with the package, is available
from references given in the paper.}

Thank you.

\textit{Quibbles and queries are:}
\begin{enumerate}
\item 
\textit{
The legend for Table 1 begins: ``Selected tree algorithms than can be
interfaced through partykit...''  Surely it is the output from selected
implementations that can be interfaced, not the algorithm used.}

Yes, thank you. We completely agree with you on the distinction between
`algorithm' and it's `implementation' and changed the caption accordingly.

\textit{The word
`algorithm' seems to me anyway inappropriate in this context.  The same (or
very similar) algorithm may have different implementations.  Different
methods with their inevitably different algorithms (including, in partykit:
classical `constant' trees, model-based trees, and conditional inference
trees) further widen the range of implementations.}

We completely agree and consistently use "algorithm" for the concept
and "implementation" for the software throughout the paper now.

\item 
\textit{
The distinction between implementation (what the authors call `algorithm')
and output is crucial also because as far as I am able to judge, partykit's
abilities are limited to taking over and processing some part only of the
output from, e.g., R's rpart package.  As I understand, there is no ability
to use or process, e.g., results from rpart's on the fly cross-validation.}

Yes, this is right. It is not possible to directly work on the rpart
results.  The reason is that we coerce rpart objects to constparty objects
and this class does not capture cost-complexity pruning since not all
algorithms have a notion of a risk function and thus perform cost-complexity
pruning.  The tools provided by partykit require that pruning is performed
by rpart or J4.8 or by user-implemented add-ons.

\textit{Would it be straightforward to incorporate structure that is able to
accommodate such information?  See also point 3 that now follows.}

Yes, one `just' needs to define a new class that inherits from constparty,
possibly containing a cost-complexity table in the info slot with
appropriate methods for pruning (the nodeprune function can be used).  An
example on the use of nodeprune is given in the constparty vignette,
Section~3, and mob has a novel internal pruning functionality.

\item 
\textit{
I note the comment on the help page for the partykit function ctree() that
the inference process is based on permutation tests, with one or other type
of adjustment for multiple testing.  Then follows the strong claim that:
This statistical approach ensures that the right sized tree is grown and no
form of pruning or cross-validation or whatsoever is needed.  I think it
pertinent to ask what has been done to judge the effectiveness of this
approach, assessed using separate test data, against the approach used in,
e.g., rpart or CART?  This is an issue for the help page, rather than
perhaps for the paper.}

This is a somewhat philosphical issue, discussed in more detail in
Section~3.4 of \cite{Hothorn+Hornik+Zeileis:2006}. If one grows a tree
on a very large data set with nominal level of 5\%, many statistically
significant but practically irrelevant splits will be found. For small and
medium sized problems, conditional inference trees with nominal level of 5\% 
and appropriate adjustment for multiplicity were shown to produce trees
of the correct size \citep[simulation results
in][]{Hothorn+Hornik+Zeileis:2006}.

\item 
\textit{
How well-adapted are the tools in partykit for implementing random forest
or other ensemble type approaches? I note that partykit does not (yet?) have
an equivalent of the cforest() function that is in the older party package.
}

In principle it is straightforward to set-up a class for forests as a list
of partynode objects. The advantage is that only the tree
structures vary across the ensemble but not the data. However, we found such
a class of rather limited use because we didn't see a unified way to produce
meaningful plots (variable importances, partial dependencies etc) for
ensembles obtained by different forest algorithms yet. Implementing a new
class for forests (maybe called parties or orgie) is on our to do list.

\end{enumerate}

\textit{
At the very least, I would like to see brief comments on where splitting and
accuracy assessment strategies, short of use of separate test data, fit in
or around the toolkit.}  

The general infrastructure provided by partykit covers the representation of
trees but not tree induction itself. This makes sense because the details of
how splits are searched for and selected differ considerably between
algorithms. The same is true with respect to finding the right-sized tree.

The implementing of tree algorithms is very much simplified by partykit
because the programmer can concentrate on the technical details of learning
the tree while using the partykit infrastructure for representing and
further processing the resulting tree. The constparty vignette contains a
very dense example of such an implementation (Section 3). Note that a
non-trivial tree algorithm can be implemented by three small R functions
only.

We added an example showing how the partykit infrastructure can be used
to quickly implement model assessment strategies to the end of Section~3 in
the constparty vignette.

\textit{To what extent are toolkit functions independent of
the methodology used for the partykit functions ctree() and mob()?}

Completely. <Z FIXME> mob.

\textit{If necessary, some of the historical detail in the overview might be replaced
by references to discussion that appears elsewhere, in order to make way for
comment on these points.}

Please see the new example in Section~3 of the constparty vignette.
<Z FIXME>

\section*{Referee 3}

\textit{
The paper describes an R library, `partykit', that appears to be a unified
way of representing decision tree structures (or recursive partitioning
structures) in R and manipulating or visualizing them.  The authors note
that their library contains support for representing the output of trees
created with other packages, and the paper appears to function as a very
basic tutorial with links to very detailed R vignettes.  The package also
implements a few of the authors' own recursive partitioning algorithms.}

Our understanding of the reviews on an earlier version of this paper was
that the referees asked for this format, please see our answer to Referee 1.

\textit{
The project seems to have fairly minor activity, and appears to have had a
Google Summer of Code project in previous years.  The subversion repository
seems to indicate that the code is still being (somewhat) actively developed
despite a relative lack of commits between 2009 and 2013.}

The package was written by only two developers, hence sometimes longer
breaks to the new jobs or family businesses could not be avoided.
It is true that there was not much development in partykit between 2009 and
2013. During this time, two offspring packages (evtree and psychotree) where
developed instead and partykit was actively maintained. Please note that the
development of these two new packages did only bring some minor design or
implementation deficits in partykit to light. This raised our confidence in
partykit. For these reasons we feel that pure activity this is not an indicator of 
the quality of the code.

\textit{There appears to
be no bug tracker or open mailing list.}

E-mails to the maintainer address were always possible and are in fact the standard
way of communicating with CRAN package maintainers (and developers) and were 
used fairly regularly by both end-users and other package developers. 
Nevertheless, a mailing list was added now.

\textit{There is knowledge of the package
in the R ecosystem, but partykit does not seem to be as popular as other R
packages (admittedly, partykit is a niche product).}

One can argue about this statement being close to reality. As of now, the
package website lists seven reverse dependencies on CRAN, we are not
involved in three of them. Only 6\% of all CRAN packages have more 
reverse dependencies than partykit. And we hope that some of the 24 
reverse dependencies of party will switch to partykit
(only 2\% have more reverse dependencies than party) now that the reference
implementation of conditional inference trees is the one from partykit.
The ctree-publication was cited 643 times and the mob-paper 103 times 
(Google scholar), so we are confident that people will increasingly use
either ctree or mob and also hopefully learn about the partykit
infrastructure this way.

\textit{
Support for trees produced by other libraries is provided, but I am not
certain how appealing the infrastructure will be to developers who want to
represent trees from other systems.  It seems like for non-PMML formatted
trees, implementing a conversion layer to partykit objects wouldn't be any
easier or harder than other packages. Clearly, for developing new tree
algorithms, though, partykit would be a good choice.}

Our experience while implementing the original code for ctree 10 years ago
was that the infrastructure existing then had way too many
algorithm-specific things hardwired than to be of any use. While rpart
always allowed to change risk functions, doing something outside of the
CART-way of finding splits was not possible. Also printing and plotting
facilities were not able to, for example, put p-values onto inner nodes.

Our party implementation of ctree suffered the same problems, we had to work
extremely hard to somehow use the party plot methods to plot J4.8 multiway 
trees fitted by the RWeka package.

These experiences led to the partykit package. The reimplementation of
ctree, mob and the new implementations of trees in evtree and psychotree
turned out to be rather straightforward and we personally did benefit a lot
from the partykit infrastructure already. We suppose this is also true for
the developers of the NHEMOtree and quint packages.

\textit{
Section 5 is very qualitative and vague.  The authors should definitely
improve this in future revisions.  It is written, "lean data structures are
employed", "the `party' structures are very lean", and also "The recursive
node structure itself is extremely lightweight" but without any sort of
quantitative details, it's difficult to believe those statements without
diving into the code.}

Z

\textit{
In my opinion the authors could do one of these things to validate their
claims: provide proof that the data structures are minimal (with respect to
other implementations), show that very large tree structures can be easily
represented and handled by partykit, or simply leave out or significantly
shorten the discussion of speed and memory usage entirely.}

Z

Total cost of ownership

We added a demo (executable via demo("memory-speed")) that compares object
sizes and prediction speed for larger rpart and J48 trees.

\textit{
Overall, partykit is a well-designed and extensively documented package, but
I have some reservations about the project's activity level and popularity
going into the future.  The developers should consider setting up a mailing
list or bug tracker to make the project more open to new contributors.}

Based on our experience with the party package for 10 years now, the fact
that other developers started packages using partykit, and the feedback via
email and on conferences, most important the ``Workshop on Classification
and Regression Trees'' that took place in March 2014, we are more optimistic
about the popularity of partykit.

\bibliographystyle{plainnat}
\bibliography{../../../vignettes/party.bib}


\end{document}
