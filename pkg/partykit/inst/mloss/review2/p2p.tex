
\documentclass{article}
\usepackage{natbib}

\title{Rebuttal JMLR 14-133}
\author{Torsten Hothorn and Achim Zeileis}

\begin{document}

\maketitle

\section*{AE}

\textit{
I regret to inform you that your submission ``partykit: A modular toolkit
for recursive partytioning in R'' has been rejected from publication at JMLR
Track for Machine Learning Open Source Software.  Two of the reviewers of
the previous version were enlisted for this version, and one new reviewer
was added.  While the reviewers voted for ``accept'', the actual reviews
were less positive and raised several serious concerns about the project. 
If the authors choose to resubmit the project, please include a description
addressing the reviewers' concerns.}

\section*{Referee 1}

\textit{
This is a clean and concise description of an interesting set of tools for
data analysis with recursive partitioning.  It is mainly a description or
announcement, and has minimal polemic or pedagogical value.  It will be a
useful reference for those a basic outline of the software's capabilities,
and provides a useful bibliography.}

Tja.

mention: previous round of reviews asked for this format of the paper


\section*{Referee 2}

\textit{
This paper describes work that, in my understanding, does two things:
\begin{enumerate}
\item It provides infrastructure that can in principle be used with any
      recursive partitioning implementation, and that at this point 
      interfaces to several different widely used implementations, 
      additional to those directly available in R's partykit package.
\item It provides implementations of two specialized tree-based methods.
\end{enumerate}
}

yes

maybe worth mentioning: due to the flexible infrastructure, the two
specialized methods (ctree, mob) have become more flexible compared to
earlier implementations (especially the mob framework)

\textit{
The software has a useful role in providing commonality across diverse
recursive partitioning methods and implementations: for structuring printed
output, for generating predictions, and for plotting results.  Very
extensive documentation, some distributed with the package, is available
from references given in the paper.}


\textit{Quibbles and queries are:}
\begin{enumerate}
\item 
\textit{
The legend for Table 1 begins: ``Selected tree algorithms than can be
interfaced through partykit...''  Surely it is the output from selected
implementations that can be interfaced, not the algorithm used.  The word
`algorithm' seems to me anyway inappropriate in this context.  The same (or
very similar) algorithm may have different implementations.  Different
methods with their inevitably different algorithms (including, in partykit:
classical `constant' trees, model-based trees, and conditional inference
trees) further wide widen the range of implementations.}

rephrase: interfaces data structures used by different implementations of different
tree growing methods / algorithms.

\item 
\textit{
The distinction between implementation (what the authors call `algorithm')
and output is crucial also because as far as I am able to judge, partykit's
abilities are limited to taking over and processing some part only of the
output from, e.g., R's rpart package.  As I understand, there is no ability
to use or process, e.g., results from rpart's on the fly cross-validation. 
Would it be straightforward to incorporate structure that is able to
accommodate such information?  See also point 3 that now follows.}

Hm, maybe store such information in \texttt{info} slots?

\item 
\textit{
I note the comment on the help page for the partykit function ctree() that
the inference process is based on permutation tests, with one or other type
of adjustment for multiple testing.  Then follows the strong claim that:
This statistical approach ensures that the right sized tree is grown and no
form of pruning or cross-validation or whatsoever is needed.  I think it
pertinent to ask what has been done to judge the effectiveness of this
approach, assessed using separate test data, against the approach used in,
e.g., rpart or CART?  This is an issue for the help page, rather than
perhaps for the paper.}

See \citep{Hothorn+Hornik+Zeileis:2006}.

But probably also rephrase...

\item 
\textit{
How well-adapted are the tools in partykit for implementing random forest
or other ensemble type approaches? I note that partykit does not (yet?) have
an equivalent of the cforest() function that is in the older party package.
}

Work in progress.

\end{enumerate}

\textit{
At the very least, I would like to see brief comments on where splitting and
accuracy assessment strategies, short of use of separate test data, fit in
or around the toolkit.}  

infrastructure independent of such things.

\textit{To what extent are toolkit functions independent of
the methodology used for the partykit functions ctree() and mob()?}

completely.

\textit{If
necessary, some of the historical detail in the overview might be replaced
by references to discussion that appears elsewhere, in order to make way for
comment on these points.}

pfff

\section*{Referee 3}

\textit{
The paper describes an R library, `partykit', that appears to be a unified
way of representing decision tree structures (or recursive partitioning
structures) in R and manipulating or visualizing them.  The authors note
that their library contains support for representing the output of trees
created with other packages, and the paper appears to function as a very
basic tutorial with links to very detailed R vignettes.  The package also
implements a few of the authors' own recursive partitioning algorithms.}

mention: previous round of reviews asked for this format

\textit{
The project seems to have fairly minor activity, and appears to have had a
Google Summer of Code project in previous years.  The subversion repository
seems to indicate that the code is still being (somewhat) actively developed
despite a relative lack of commits between 2009 and 2013.}

not a big software project, just two main developers, hence sometimes longer breaks

we feel that this is not an indicator of the quality of the code, though

\textit{There appears to
be no bug tracker or open mailing list.}

E-mails to the maintainer address were always possible and used fairly regularly
by both end-users and other package developers. A mailing list was added now.

\textit{There is knowledge of the package
in the R ecosystem, but partykit does not seem to be as popular as other R
packages (admittedly, partykit is a niche product).}

Hard to judge, but reverse depends: NHEMOtree, quint. Number of citations of party paper.

\textit{
Support for trees produced by other libraries is provided, but I am not
certain how appealing the infrastructure will be to developers who want to
represent trees from other systems.  It seems like for non-PMML formatted
trees, implementing a conversion layer to partykit objects wouldn't be any
easier or harder than other packages.}

depends, eh?

examples: CTree and display of p-values, J4.8 and multiway splits

\textit{
Clearly, for developing new tree algorithms, though, partykit would be a good choice.}

We actually have proof!

\textit{
Section 5 is very qualitative and vague.  The authors should definitely
improve this in future revisions.  It is written, "lean data structures are
employed", "the `party' structures are very lean", and also "The recursive
node structure itself is extremely lightweight" but without any sort of
quantitative details, it's difficult to believe those statements without
diving into the code.}

schwierig.

\textit{
In my opinion the authors could do one of these things to validate their
claims: provide proof that the data structures are minimal (with respect to
other implementations), show that very large tree structures can be easily
represented and handled by partykit, or simply leave out or significantly
shorten the discussion of speed and memory usage entirely.}

schwierig.

\textit{
Overall, partykit is a well-designed and extensively documented package, but
I have some reservations about the project's activity level and popularity
going into the future.  The developers should consider setting up a mailing
list or bug tracker to make the project more open to new contributors.}

done.

\bibliographystyle{plainnat}
\bibliography{../../../vignettes/party.bib}


\end{document}
