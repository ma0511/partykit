
\documentclass{article}
\usepackage{a4wide}
\usepackage[authoryear,round]{natbib}

\setlength{\parskip}{0.5ex plus0.1ex minus0.1ex}
\setlength{\parindent}{0em}

\title{JMLR-14-133-4: Point-to-Point Reply}
\author{Torsten Hothorn and Achim Zeileis}

\begin{document}

\maketitle

\section*{AE}

\textbf{\textit{%
I am concerned that your new revision is essentially a rebuttal of the
request of the reviewer about adding random forest functionality (which
existed in the previous version) into the current version.  As I said in my
previous decision, porting functionality of party to partykit is important,
and in my opinion a reasonable request.
}}

\smallskip

We now provide a reimplementation of \texttt{party::cforest} in the new
\textbf{partykit} framework.  From the reviews we were not entirely sure
what your desired solution for the reimplementation was: We decided to write
a new high-level implementation in R, demonstrating the breadth and
usefulness of the \textbf{partykit} infrastructure also for ensemble
learning.


\section*{Referee 3}

\textbf{\textit{%
The paper itself is fine.  Last review I had no further comments on the
paper itself.  My review here then is restricted to the point-by-point
reply.  This "discussion" is going downhill quickly and I will thus keep my
comments short.
The authors have pointed out some measures of project activity, which are
fine.  Of my concerns, the activity level and lack of open mailing list are
less of a big deal, and my review stated as much.
The remaining sticking point is this:
"I am not convinced that the breadth of this contribution is sufficient." (from the previous review)
The author's replies have not convinced me otherwise despite their extensive
length.  My suggestion was that support for forests be added, or some other
support be added to increase the utility or breadth of partykit.  The
responses given to this idea by the authors are concluded with remarks like
the following:
\begin{itemize}
\item "These [ideas] deserve further thought and work, both conceptually and computationally."
\item "Therefore, we would much rather follow Reviewer 2 and devote more future work to this important topic."
\item "This is on our to-do list, though."
\end{itemize}
If these things are on the authors' to-do lists, then they should do them
and resubmit when they are done.}}

\smallskip

The new \texttt{partykit::cforest} shares the benefits of
\texttt{party::cforest} over \texttt{randomForest::randomForest} (such as surrogate
splits for dealing with missing values in the predictor variables or the
ability of handling right-censored data) and even has a couple of
advantages over the old \texttt{party::cforest}: First, it is very flexible
and can exploit some features readily available in R, e.g., parallelization
on multicore machines or clusters of workstations.  Second, it can be used
as a blueprint for users/developers who wish to create custom forests with
their desired tree learners.  However, as \texttt{partykit::cforest} is
written entirely in R, it is slower than the old \texttt{party::cforest}
when running on a single core only. Third, quantile regression forests
(Meinshausen, JMLR 2006) are now part of the framework and even extensions
to density estimation are available (see \texttt{?cforest} for an example).

\textbf{\textit{%
Personally I am not willing to consider acceptance as-is with the assumption
that these features will be added later because the authors have stated
elsewhere that partykit is mature and now only requires maintenance.
The package would be stronger and relevant to a wider audience, even if the
added support was only preliminary.  And according to the authors
themselves, this would be a significant contribution because flexible
support for forests is a "real challenge".
}}

\smallskip

The new \texttt{partykit::cforest} basically implements old functionality
using the new framework. The code-base is much denser now and the two
implementations were tested to return identical results (in situations where
is can be expected). It also allows a little more flexibility with respect
to probabilistic forecasts (via quantile regression forests) but is still
limited to trees with constant fits in the nodes. Therefore, the challenge
of adding flexible support for forests of model-based is still open and
definitively a research project in its own right because also the underlying
theory has to be developed.

\end{document}
