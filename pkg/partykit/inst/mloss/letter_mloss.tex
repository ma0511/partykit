\documentclass[ngerman,uzh]{scrlttr2}
\usepackage[T1]{fontenc}
\usepackage[utf8]{inputenc}

%% ANPASSUNGSMÖGLICHKEITEN
\KOMAoptions{%
%backaddress=false%  Keine Rücksendeadresse im Adressfeld
%,foldmarks=false%  Faltmarken ausschalten
}

\usepackage{babel}


\begin{document}

%% VARIABLEN
% Falls Unterschrift vom definierten Namen abweicht:
\setkomavar{signature}{Torsten Hothorn}
\setkomavar{subject}{Betreff}
%\setkomavar{specialmail}{Interne Post}

\begin{letter}{
Drs.~Murphy and Schölkopf \\
Editors-in-chief \\
Journal of Machine Learning Research}

\subject{JMLR-MLOSS submission}

\opening{Dear Drs.~Murphy and Schölkopf,}

my co-author Achim Zeileis and I would like to submit the manuscript
``\textbf{partykit}: An Open-Source Toolkit for Recursive Partytioning in R'' to the
Machine Learning Open-Source Software track of the Journal of Machine
Learning Research.

The \textsf{R} add-on package \textbf{partykit} described in this submission implements a
flexible toolkit for learning, representing, summarizing, and visualizing a
wide range of tree-structured models.  The package offers a basic class
infrastructure for representing fitted trees, a unified set of methods for
printing and plotting of trees as well as for computing predictions for new
observations, tailored functionality for trees with constant fits and with
parametric models in each terminal node, and reimplementations of conditional
inference trees (CTree) and model-based recursive partitioning (MOB).

This manuscript is related to JMLR-07-240-1 ``Let's Have a party! An
Open-Source Toolbox for Recursive Partytioning'' describing the
\textbf{party} add-on
package.  This manuscript was rejected with encouragement for resubmission by
JMLR-MLOSS editor Cheng Soon Ong in November 2007.  The main concerns of the
reviewers were that it was not easy to extend the old \textbf{party} package by adding
new tree learners or representing existing trees in our package.  This was
a very important suggestion and part of the motivation for starting from
scratch and develop a new R package (\textbf{partykit}) that easily allows to
accommodate our own tree algorithms but makes the underlying infrastructure
easily accessible to others. Actually, the \textbf{partykit} package in its current
form goes well beyond what was suggested in the reviews of JMLR-07-240-1
and both manuscript and software were entirely rewritten. Hence, we felt it
to be more appropriate to submit this as a new manuscript rather than a
revision of JMLR-07-240-1.  However, if you would like to see a formal
point-to-point reply to the old submission, please let us know.

Sincerely yours, \\
\includegraphics[width = .3\textwidth]{TH_Unterschrift}

\end{letter}

\end{document}