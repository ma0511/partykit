\documentclass[10pt]{article}

\usepackage{url}
\usepackage[authoryear,round]{natbib}

\renewcommand{\title}[1]{\begin{center}{\bf \LARGE #1}\end{center}}
\newcommand{\affiliations}{\footnotesize}
\newcommand{\keywords}{\paragraph{Keywords:}}

\setlength{\oddsidemargin}{0cm} \setlength{\evensidemargin}{0cm}
\setlength{\textwidth}{16.5cm} \setlength{\topmargin}{-1cm}
\setlength{\textheight}{24.5cm}

\let\code=\texttt
\let\pkg=\textbf
\let\proglang=\textsf
\newcommand{\class}[1]{`\code{#1}'}

\begin{document}
\pagestyle{empty}

\title{\pkg{partykit}: A Toolbox for Recursive Partytioning}

\begin{center}
  {\bf Torsten Hothorn$^{1}$, Achim Zeileis$^{2}$}
\end{center}

\begin{affiliations}
\begin{enumerate}
  \item Institut f\"ur Statistik, Ludwig-Maximilians-Universit\"at M\"unchen, Germany\\
        \texttt{Torsten.Hothorn@R-project.org}
  \item Department of Statistics and Mathematics, Wirtschaftsuniversit\"at Wien, Austria\\
        \texttt{Achim.Zeileis@R-project.org}
\end{enumerate}
\end{affiliations}

\keywords Regression Trees, Visualization, CTree, CHAID, Weka, PMML.

\vskip 0.8cm

Recursive partitioning methods, or simply ``trees'', are simple yet powerful
methods for capturing regression relationships. Since the publication of the
automated interaction detection (AID) algorithm in 1964, many extentions,
modifications, and new approaches have been suggested in both the statistics
and machine learning communities. Most of the standard algorithms are
available to the \proglang{R} user, e.g., through packages
\pkg{rpart} \citep{rpart},
\pkg{RWeka} \citep{RWeka},
\pkg{party} \citep{party}, or
\pkg{mvpart} \citep{mvpart}.

However, no common infrastructure is available for representing trees
fitted by different packages. Consequently, the capabilities for extraction of
information---such as predictions, printed summaries, or visualizations---vary
between packages and come with somewhat different user interfaces.
Furthermore, extensions or modifications often require considerable
programming effort, e.g., if the median instead of the mean of a numerical
response should be predicted in each leaf of an \class{rpart} tree.
Similarly, implementations of new tree algorithms might also require new
infrastructure if they have features not available in the above-mentioned
packages, e.g., multi-way splits or more complex models in the leafs.

To overcome these difficulties, the \pkg{partykit} package \citep{partykit} offers a 
unified representation of tree objects along with \code{predict()}, 
\code{print()}, and \code{plot()} methods. Trees are represented through 
a new flexible class \class{party} which can, in principle, capture
all trees mentioned above but can also accomodate multi-way or functional
splits, as well as complex models in (leaf) nodes. The package is currently
under development at R-Forge but already provides conversion methods for
trees of classes \class{rpart}, \class{J48}, and \class{pmmlTreeModel}
as well as a re-implementation of conditional inference trees \citep{Hothorn+Hornik+Zeileis:2006}.

In our presentation, we will only sketch details of these classes and 
corresponding methods and focus on applications of the toolkit including
extended visualizations for \class{rpart} or \class{J48} objects, fast predictions
on millions of new observations, and a new implementation of the classical CHAID
algorithm.

\bibliographystyle{jss}
\bibliography{party}

\end{document}
