\documentclass[10pt]{article}

\usepackage{url}
\usepackage{natbib}

\renewcommand{\title}[1]{\begin{center}{\bf \LARGE #1}\end{center}}
\newcommand{\affiliations}{\footnotesize}
\newcommand{\keywords}{\paragraph{Keywords:}}

\setlength{\oddsidemargin}{0cm} \setlength{\evensidemargin}{0cm}
\setlength{\textwidth}{16.5cm} \setlength{\topmargin}{-1cm}
\setlength{\textheight}{24.5cm}

\begin{document}
\pagestyle{empty}

\title{\textbf{partykit}: A Toolbox for Recursive Partytioning}

\begin{center}
  {\bf Torsten Hothorn$^{1}$ and Achim Zeileis$^{2}$}
\end{center}

\begin{affiliations}
1. Institut f\"ur Statistik, Ludwig-Maximilians-Universit\"at M\"unchen \par
2. Department of Statistics and Mathematics, Wirtschaftsuniversit\"at Wien \par
\texttt{\{Torsten.Hothorn, Achim.Zeileis\}@R-project.org}
\end{affiliations}

\keywords Regression Trees, Visualization, CTree, CHAID, Weka, PMML.

\vskip 0.8cm

Recursive Partitioning methods, or simply `trees', are simple yet powerful
methods to estimate regression relationships. Since the publication of the
Automated Interaction Detection (AID) algorithm in 1964, many extentions,
modifications and new approaches have been suggested in both the statistics
and machine learning communities. Most of the algorithms in use today are
available to the \textsf{R} user, for example from packages \textbf{rpart}
\citep{PKG:rpart},
\textbf{RWeka} \citep{PKG:RWeka}, \textbf{party} \citep{PKG:party}, or \textbf{mvpart}
\citep{PKG:mvpart}.

Computations on fitted trees or the implementation of a new or modified
recursive partitioning method, however, are relatively difficult tasks.
As an example, one might want to predict a numeric response by the median
instead of the mean based on an \textit{rpart} object. Or, instead of
binary splits, one is going to implement an algorithm which potentially
uses multiway splits, maybe containing a non-trivial model in each terminal
node. These, and other, tasks require extensive programming,
mainly because each package dealing with (some specialized form of)
recursive partitioning uses a different internal representation of a tree. Thus, no
standard user-interface exists, and methods for predictions, graphical or textual
visualizations, or other computations heavily rely on these internals.

To overcome these difficulties, the \textbf{partykit} package \citep{PKG:partykit} offers a 
unified representation of tree objects along with \texttt{predict}, 
\texttt{print}, and \texttt{plot} methods. Trees are represented through 
a class \textit{party}, with building blocks \textit{partynode} and 
\textit{partysplit}. The design of these classes covers splits in arbitrary
functions of (one or more) variables, including multiway splits. In
addition, non-constant fits in the terminal nodes are supported as well.
Conversion methods exist for the most important sources of recursive
partitioning objects, currently \textit{rpart}, \textit{J48} and
\textit{pmmlTreeModel} objects can easily be converted into a \textit{party}
object to compute with. The \textbf{partykit} also offers a re-implementation
of conditional inference trees \cite{Hothorn:2006:JCGS}, illustrating the usefulness of the generic
infrastructure.

In our presentation, we will only sketch details of these classes and 
corresponding methods and focus
on applications of the toolkit, for example for producing nice and
non-standard plots
for \textit{rpart} or \textit{J48} objects, for computing predictions
on millions of new observations really fast, and for implementing CHAID, 
a popular classification tree algorithm.

\bibliographystyle{plainnat}
\bibliography{partykit}

\end{document}
